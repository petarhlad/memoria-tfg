\chapter{Introducció}

Els expenedors automàtics de begudes fredes tenen una vida útil bastant llarga però les compres amb efectiu s'estan quedant enrere.

Pels expenedors que originalment no inclouen sistemes de pagament sense monedes existeixen accessoris que se'ls pot afegir com acceptadors de bitllets o de targetes de crèdit.

En aquest projecte el que intentarem és crear una solució que sigui integrable en un expenedor existent que no inclou sistemes de pagament que no sigui amb monedes. La idea és util per integrar en expenedors que es troben en l'edifici d'una empresa i es vol, per exemple, donar descomptes als seus empleats en funció del consum o de l'hora del dia.

\section{Objectius del projecte}

L'objectiu d'aquest projecte és dissenyar i desenvolupar un prototip per a un sistema \textit{cashless} que es pugui integrar dins d'expenedors automàtics existents. El projecte es centrarà en un model específic d'expenedor automàtic de begudes fredes (Dixie-Narco DNCB 386) perquè és l'expenedor que hi ha disponible per al desenvolupament.

Els principals objectius es poden resumir en els següents punts:

\vspace{-0.5em}
\begin{description}[font=\normalfont\textsl]
\item [Dissenyar i desenvolupar l'aplicació de servidor. ] El sistema de la solució serà un sistema centralitzat. Tot es gestionarà des d'un mateix lloc, i aquest serà l'aplicació de servidor.
\item [Dissenyar i desenvolupar l'aplicació de client. ] Per tal de poder controlar l'expenedor automàtic i comunicar-se amb l'aplicació de servidor, és necessaria l'aplicació de client.
\item [Dissenyar  Desenvolupar el sistema de hardware. ] Per tal de que l'aplicació de client pugui controlar l'expenedor, és necessari el sistema de hardware que farà d'interfície entre l'aplicació i els circuits de l'expenedor.
\end{description}

La feina feta durant aquests mesos s'ha centrat en desenvolupar el codi de les aplicacions i en desenvolupar el desplegament de hardware que s'integrarà a l'expenedor.

\section{Planificació del temps}

La planificació temporal que s'ha seguit durant el projecta es mostra a la Figura~\ref{fig:gantt} Diagrama de Gantt.

La feina s'ha repartit en 6 blocs. La descripció detallada de cada bloc es troba a l'Apèndix~\ref{app:gantt}.

\begin{figure}[ht]
\center
\begin{ganttchart}[
hgrid,
bar/.append style={fill=blue!50},
vgrid={*4{dotted},*1{dashed},*3{dotted},*1{dashed},*3{dotted},*1{dashed},*3{dotted},*1{dashed},*4{dotted},*1{dashed}},
x unit=0.47cm,
title height=1, 
y unit title=0.6cm,
y unit chart=0.8cm]{1}{22}
\gantttitle{Project timeline in months/weeks}{22} \\
\gantttitle{FEBR}{4}
\gantttitle{MARCH}{5}\gantttitle{APRIL}{4}\gantttitle{MAY}{4}\gantttitle{JUNE}{5} \\
\gantttitle[title label node/.append style={below left=-8pt and -7pt}]{\footnotesize\textit{Week}\quad1}{1}
\gantttitlelist
{2,...,22}{1} \\ 
\ganttbar{Training}{1}{6} \\
\ganttbar{System Design}{5}{8}\\
\ganttbar{System Development}{9}{17} \\ 
\ganttbar{Test Unit Design}{18}{18}\\ 
\ganttbar{Test Unit Development}{19}{20}\\
\ganttbar{Writing}{21}{22}
\end{ganttchart}
\caption{Diagrama de Gantt del projecte}
\label{fig:gantt}
\end{figure}








