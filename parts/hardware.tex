\chapter{Disseny de hardware}\label{chapter:disseny de hardware}
En aquest capítol comentarem tot el disseny del hardware que s'ha seguit per poder controlar l'expenedor automàtic assolint els requisits definits anteriorment.

\section{Disseny elèctric}
A partir de l'esquema elèctric simplificat de l'expenedor, la idea del nostre disseny elèctric és d'afegir un circuit paral·lel a la màquina de canvi i als botons de selecció i que pugui deshabilitar la màquina de canvi durant el transcurs d'una venda en el nostre sistema. 

La idea, doncs, és afegir un relé que fa de commutador abans de la màquina de canvi que derivi el corrent cap.



\section{Components}
Com a controlador central de l'expenedor es farà servir un ordinador de placa petita \textit{RaspberryPi 2 B} (a partir d'ara, RPi), ja que disposa de bastants pins d'entrada/sortida, se li pot instal·lar un sistema operatiu basat en linux (que fa molt més fàcil el disseny de l'aplicació de client), té connexió a internet, possibilitat de connectar-hi una pantalla per HDMI o bé una pantalla tàctil.

Per a la interfície d'usuari usarem la pantalla tàctil \textit{Raspberry Pi Touch 7"}.

Per a poder accionar els circuits que accionen els motors de l'expenedor des de la RPi farem servir una placa de relés activats per nivell baix.



RPi necessitarà una alimentació de 5V i 1A.
La pantallà tàctil necessita una alimentació de 5V i 1A.
La placa de relés necessita una alimentació de 12V.
