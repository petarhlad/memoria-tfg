\chapter{Conclusions i tasques futures}

En general, en aquest projecte, un cop plantejat el problema a solucionar, el que més temps s'ha hagut d'invertir ha estat l'aprenentatge de com funciona l'expenedor i l'aprenentatge de les eines de programació.

De cara als objectius podem dir que s'ha elaborat un prototip funcional però que encara falta per mecanitzar-lo dins de l'expenedor.

De tasques futures que s'hauran de realitzar sobre el projecte una és la mecanització definitiva del sistema i una altra és el \textit{deploy} de l'aplicació en un servidor. L'aplicació de servidor ha estat pensada de manera que sigui fàcil d'integrar en una aplicació Django existent amb base de dades d'usuaris. De fet, l'objectiu de futur d'aquest projecte és integrar l'aplicació de servidor en la intranet d'usuaris de l'associació on tenim l'expenedor de manera que en puguin gaudir tots els estudiants de l'associació.

Una altra cosa que també s'hauria de fer és desenvolupar unes interfícies personalitzades a l'aplicació de servidor per tal de facilitar l'entrada de nous usuaris o bé la recàrrega de saldo. Donat que la nova generació de telèfons mòbils incorpora lector NFC, ses planteja desenvolupar una aplicació per Android a part de les interfícies per a l'aplicació de servidor.